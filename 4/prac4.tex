\documentclass[14pt,a4paper]{extarticle}
\usepackage{../templates/preamble}

\newcommand{\reportof}{практической работе №4}
\newcommand{\theme}{Методы нахождения производной.}

\begin{document}
\input{../templates/titlepage.tex}

\section*{Цель работы}
        Изучить основные методы нахождения производных.


\section*{Отчёт о проделанной работе}
        В первом задании необходимо найти промежутки возрастания и
убывания нескольких функций с помощью их производных. Для этого воспользуемся
правилами нахождения производных. Воспользовавшись кодом из предыдущего
практического задания, узнаем промежутки знакопостоянства и нули
производной и построим графики функций. Код приведён на рисунке \ref{fig:4.1-code}, результаты
--- на рисунке \ref{fig:4.1-result}.

\begin{figure}[h!]
    \centering
    \includegraphics[width=0.7\linewidth]{figures/1-code.png}
    \caption{Расчёт промежутков знакопостоянства и отображение функций}
    \label{fig:4.1-code}
\end{figure}

\newpage

\begin{figure}[ht!]
    \centering
    \begin{subfigure}{.333\textwidth}
        \centering
        \includegraphics[width=\linewidth]{figures/1-1.png}
    \end{subfigure}%
    \begin{subfigure}{.333\textwidth}
        \centering
        \includegraphics[width=\linewidth]{figures/1-2.png}
    \end{subfigure}%
    \begin{subfigure}{.333\textwidth}
        \centering
        \includegraphics[width=\linewidth]{figures/1-3.png}
    \end{subfigure}%
    \caption{Результаты задания 4.1}
    \label{fig:4.1-result}
\end{figure}


Во второй части необходимо найти производные функций. Воспользуемся для этого
функционалом $sympy$, а именно --- функцией \textit{diff}, которая позволяет найти
производную выражения относительно переменной. Результаты задания 4.2 см.рис. \ref{fig:4.2},
результаты задания 4.3 --- рис. \ref{fig:4.3}.


\begin{figure}[ht!]
    \centering
    \begin{subfigure}{.5\textwidth}
        \centering
        \includegraphics[width=0.7\linewidth]{figures/2-1.png}
    \end{subfigure}%
    \begin{subfigure}{.5\textwidth}
        \centering
        \includegraphics[width=0.7\linewidth]{figures/2-2.png}
    \end{subfigure}%
    \caption{Результаты задания 4.2}
    \label{fig:4.2}
\end{figure}
\newpage
\begin{figure}[ht!]
    \centering
    \begin{subfigure}{.5\textwidth}
        \centering
        \includegraphics[width=0.7\linewidth]{figures/3-1.png}
    \end{subfigure}%
    \begin{subfigure}{.5\textwidth}
        \centering
        \includegraphics[width=0.7\linewidth]{figures/3-2.png}
    \end{subfigure}%
    \caption{Результаты задания 4.3}
    \label{fig:4.3}
\end{figure}


\section*{Вывод}

        В ходе выполнения работы я вспомнил правила нахождения производных,
познакомился с символическим поиском производной с помощью \textit{sympy}.

\end{document}